%%% Glossar erstellen (Package Glossaries)
%
% Syntaxinfo - Glossareintrag:
% \newglossaryentry{<label>}{<settings>}
% 
% <label> setzt die spezifische Abkürzung nach der im Text referenziert werden kann (z.B. mit \gls{<label>})
%
% mögliche <settings>:
% name = {<text>}    setzt den Namen der im Text als Abkürzung angezeigt wird
% description = {<text>}    Ist die Erklärung der Abkürzung
% sort = {<text>}    wenn bei der Sortierung des Glossars dieser Text statt der Abkürzung verwendet werden soll
% plural = {<text>}    Setzt die Abkürzung mit einem spezifischen Plural
% symbol = {<text>}    Definiert ein bestimmtes Symbol für diese Abkürzung
%
% ------------------------------
% Syntaxinfo - Akronym:
% \newacronym{<label>}{<abbrv>}{<full>}
%
% <label> setzt die spezifische Abkürzung nach der im Text referenziert werden kann (z.B. mit \gls{<label>})
% <abbrv> setzt den Namen der im Text als Abkürzung angezeigt wird
% <full> Volle Ausschreibung der Abkürzung
%
% Referenzierung
% \gls{<label>}        Standardreferenzierung fuer Glossar und Akronyme
% \glspl{<label>}     Pluralreferenzierung
% \Gls{<label>}       Referenzierung mit großem Anfangsbuchstaben
% \Glspl{<label>}    Pluralreferenzierung mit großem Anfangsbuchstaben
% \acrlong{<label>} Akronym Referenzierung kurz
% \acrshort{<label>}Akronym Referenzierung ausgeschrieben
% \acrfull{<label>}    Akronym volle ausschreibung mit abkuerzung in klammern
%
%
% Example:
%
%\newacronym{html}{HTML}{Hypertext Markup Language}
%
%\newglossaryentry{wysiwyg}
%{
%    name={WYSIWYG-Editor},
%    description={is a text editing program in which the written text is directly shown in the way it is supposed to look in a printed or browser computed version.}
%}
%\newglossaryentry{BEG}
%{
%    name={Bosch Engineering GmbH},
%    description={is a text editing program in which the written text is directly shown in the way it is supposed to look in a printed or browser computed version.}
%}
\abrgls{beg}{BEG}{Bosch Engineering GmbH}{is a text editing program in which the written text is directly shown in the way it is supposed to look in a printed or browser computed version.}
%\newacronym{BEG}{BEG}{Bosch Engineering GmbH}
%\newacrgls{html}{Hypertext Markup Language}{text zur beschreibung}